% Options for packages loaded elsewhere
\PassOptionsToPackage{unicode}{hyperref}
\PassOptionsToPackage{hyphens}{url}
\PassOptionsToPackage{dvipsnames,svgnames,x11names}{xcolor}
%
\documentclass[
  letterpaper,
  DIV=11,
  numbers=noendperiod]{scrartcl}

\usepackage{amsmath,amssymb}
\usepackage{iftex}
\ifPDFTeX
  \usepackage[T1]{fontenc}
  \usepackage[utf8]{inputenc}
  \usepackage{textcomp} % provide euro and other symbols
\else % if luatex or xetex
  \usepackage{unicode-math}
  \defaultfontfeatures{Scale=MatchLowercase}
  \defaultfontfeatures[\rmfamily]{Ligatures=TeX,Scale=1}
\fi
\usepackage{lmodern}
\ifPDFTeX\else  
    % xetex/luatex font selection
\fi
% Use upquote if available, for straight quotes in verbatim environments
\IfFileExists{upquote.sty}{\usepackage{upquote}}{}
\IfFileExists{microtype.sty}{% use microtype if available
  \usepackage[]{microtype}
  \UseMicrotypeSet[protrusion]{basicmath} % disable protrusion for tt fonts
}{}
\makeatletter
\@ifundefined{KOMAClassName}{% if non-KOMA class
  \IfFileExists{parskip.sty}{%
    \usepackage{parskip}
  }{% else
    \setlength{\parindent}{0pt}
    \setlength{\parskip}{6pt plus 2pt minus 1pt}}
}{% if KOMA class
  \KOMAoptions{parskip=half}}
\makeatother
\usepackage{xcolor}
\setlength{\emergencystretch}{3em} % prevent overfull lines
\setcounter{secnumdepth}{5}
% Make \paragraph and \subparagraph free-standing
\ifx\paragraph\undefined\else
  \let\oldparagraph\paragraph
  \renewcommand{\paragraph}[1]{\oldparagraph{#1}\mbox{}}
\fi
\ifx\subparagraph\undefined\else
  \let\oldsubparagraph\subparagraph
  \renewcommand{\subparagraph}[1]{\oldsubparagraph{#1}\mbox{}}
\fi


\providecommand{\tightlist}{%
  \setlength{\itemsep}{0pt}\setlength{\parskip}{0pt}}\usepackage{longtable,booktabs,array}
\usepackage{calc} % for calculating minipage widths
% Correct order of tables after \paragraph or \subparagraph
\usepackage{etoolbox}
\makeatletter
\patchcmd\longtable{\par}{\if@noskipsec\mbox{}\fi\par}{}{}
\makeatother
% Allow footnotes in longtable head/foot
\IfFileExists{footnotehyper.sty}{\usepackage{footnotehyper}}{\usepackage{footnote}}
\makesavenoteenv{longtable}
\usepackage{graphicx}
\makeatletter
\def\maxwidth{\ifdim\Gin@nat@width>\linewidth\linewidth\else\Gin@nat@width\fi}
\def\maxheight{\ifdim\Gin@nat@height>\textheight\textheight\else\Gin@nat@height\fi}
\makeatother
% Scale images if necessary, so that they will not overflow the page
% margins by default, and it is still possible to overwrite the defaults
% using explicit options in \includegraphics[width, height, ...]{}
\setkeys{Gin}{width=\maxwidth,height=\maxheight,keepaspectratio}
% Set default figure placement to htbp
\makeatletter
\def\fps@figure{htbp}
\makeatother

\KOMAoption{captions}{tableheading}
\makeatletter
\@ifpackageloaded{caption}{}{\usepackage{caption}}
\AtBeginDocument{%
\ifdefined\contentsname
  \renewcommand*\contentsname{Table of contents}
\else
  \newcommand\contentsname{Table of contents}
\fi
\ifdefined\listfigurename
  \renewcommand*\listfigurename{List of Figures}
\else
  \newcommand\listfigurename{List of Figures}
\fi
\ifdefined\listtablename
  \renewcommand*\listtablename{List of Tables}
\else
  \newcommand\listtablename{List of Tables}
\fi
\ifdefined\figurename
  \renewcommand*\figurename{Figure}
\else
  \newcommand\figurename{Figure}
\fi
\ifdefined\tablename
  \renewcommand*\tablename{Table}
\else
  \newcommand\tablename{Table}
\fi
}
\@ifpackageloaded{float}{}{\usepackage{float}}
\floatstyle{ruled}
\@ifundefined{c@chapter}{\newfloat{codelisting}{h}{lop}}{\newfloat{codelisting}{h}{lop}[chapter]}
\floatname{codelisting}{Listing}
\newcommand*\listoflistings{\listof{codelisting}{List of Listings}}
\makeatother
\makeatletter
\makeatother
\makeatletter
\@ifpackageloaded{caption}{}{\usepackage{caption}}
\@ifpackageloaded{subcaption}{}{\usepackage{subcaption}}
\makeatother
\ifLuaTeX
  \usepackage{selnolig}  % disable illegal ligatures
\fi
\usepackage{bookmark}

\IfFileExists{xurl.sty}{\usepackage{xurl}}{} % add URL line breaks if available
\urlstyle{same} % disable monospaced font for URLs
\hypersetup{
  pdftitle={Estimation of the effect of climate on infectious disesaes},
  colorlinks=true,
  linkcolor={blue},
  filecolor={Maroon},
  citecolor={Blue},
  urlcolor={Blue},
  pdfcreator={LaTeX via pandoc}}

\title{Estimation of the effect of climate on infectious disesaes}
\author{}
\date{}

\begin{document}
\maketitle

\renewcommand*\contentsname{Table of contents}
{
\hypersetup{linkcolor=}
\setcounter{tocdepth}{3}
\tableofcontents
}
\section{Motivating Question}\label{motivating-question}

If climate change (through some sort of variable such as temperature or
precipitation) is affecting the number of cases of infectious diseasse,
it is an outstanding question how strong this effect must be to identify
it from some background autocorrelated value.

How strong does the signal of climate change need to be to detect it?

\section{Data \& model}\label{data-model}

Assuming we have some data on observed cases of a given infectious
disease. The relationship between those observed cases and actual cases
is a state process with some observation error, \(\epsilon_o\). Cases
themselves are now given as a state space model where the number of
cases at time \(t+1\) are driven by the effect of both temperature
variance (consistent through time) and mean temperature (increasing
through time), as well as an unobserved driver that is correlated
through time with the mean temperature.

Let: - \(Y_t\) represent the true number of cases at time \(t\). -
\(Y_t^{\text{obs}}\) represent the observed number of cases at time
\(t\), which includes observational error \(\epsilon_o\). - \(X_{\mu}\)
represent the mean temperature at time \(t\), and let \(X_{\sigma}\)
represent the temperature variance, assumed constant over time. -
\(U_t\) represent the unobserved driver correlated with mean temperature
\(X_{\mu}\).

\subsection{State Process (Evolution)
Equation}\label{state-process-evolution-equation}

For the true cases \(Y_t\), we specify that the cases at time \(t+1\)
are driven by the temperature effects and an unobserved driver \(U_t\)
as follows:

\begin{equation}
Y_{t+1} \mid Y_t, X_{\mu}, U_t \sim \text{Poisson} \left(Y_t \exp \left( \alpha + \beta_{\mu} X_{\mu} + \beta_{\sigma} X_{\sigma} + \gamma U_t \right)\right)
\end{equation}

where: - \(\alpha\) is an intercept term. - \(\beta_{\mu}\) is the
effect of mean temperature \(X_{\mu}\) on the cases. -
\(\beta_{\sigma}\) is the effect of temperature variance \(X_{\sigma}\).
- \(\gamma\) captures the effect of the unobserved driver \(U_t\) on the
cases.

\subsection{Latent Driver Process}\label{latent-driver-process}

We assume that the unobserved driver \(U_t\) has temporal correlation
and is influenced by the mean temperature:

\begin{equation}
U_{t+1} \mid U_t, X_{\mu} \sim \mathcal{N}(\phi U_t + \eta X_{\mu}, \sigma_U^2)
\end{equation}

where: - \(\phi\) is an autoregressive parameter governing the temporal
correlation of \(U_t\). - \(\eta\) is the strength of the correlation
between \(U_t\) and \(X_{\mu}\). - \(\sigma_U^2\) is the variance of
\(U_t\).

\subsection{Observation Equation}\label{observation-equation}

The observed cases \(Y_t^{\text{obs}}\) are related to the true cases
\(Y_t\) with observation error \(\epsilon_o\):

\begin{equation}
Y_t^{\text{obs}} \mid Y_t \sim \text{Poisson}(Y_t e^{\epsilon_o})
\end{equation}

with \(\epsilon_o \sim \mathcal{N}(0, \sigma_o^2)\), where
\(\sigma_o^2\) is the variance of the observation error.

\subsection{Priors}\label{priors}

The priors for the parameters could be specified as follows:

\begin{subequations}
\begin{align}
\alpha &\sim \mathcal{N}(0, 10) \\ 
\beta_{\mu} &\sim \mathcal{N}(0, 1) \\
\beta_{\sigma} &\sim \mathcal{N}(0, 1) \\ 
\gamma &\sim \mathcal{N}(0, 1)
\end{align}
\end{subequations}

\begin{subequations}
\begin{align}
\phi &\sim \text{Uniform}(-1, 1) \\ 
\eta &\sim \mathcal{N}(0, 1)  \\ 
\sigma_U^2 &\sim \text{Inverse-Gamma}(2, 1) \\
\sigma_o^2 &\sim \text{Inverse-Gamma}(2, 1) 
\end{align}
\end{subequations}

\subsection{Full Conditional
Derivations}\label{full-conditional-derivations}

\subsubsection{\texorpdfstring{1. Full Conditional for
\(Y_t\)}{1. Full Conditional for Y\_t}}\label{full-conditional-for-y_t}

The likelihood for \(Y_t\) comes from the Poisson model and the
observation model:

\begin{equation}
Y_{t+1} \mid Y_t, X_{\mu}, U_t \sim \text{Poisson}\left( Y_t \exp(\alpha + \beta_{\mu} X_{\mu} + \beta_{\sigma} X_{\sigma} + \gamma U_t) \right)
\end{equation}

\begin{equation}
Y_t^{\text{obs}} \mid Y_t \sim \text{Poisson}(Y_t e^{\epsilon_o})
\end{equation}

The joint likelihood is:

\begin{equation}
\begin{split}
p(Y_t \mid Y_{t+1}, Y_t^{\text{obs}}, \alpha, \beta_{\mu}, \beta_{\sigma}, \gamma, U_t, \epsilon_o) & \propto \text{Poisson}(Y_{t+1} \mid Y_t \exp(\alpha + \beta_{\mu} X_{\mu} + \beta_{\sigma} X_{\sigma} + \gamma U_t)) \times \\ 
& \text{Poisson}(Y_t^{\text{obs}} \mid Y_t e^{\epsilon_o})
\end{split}
\end{equation}

Simplifying this:

\begin{equation}
\begin{split}
p(Y_t \mid Y_{t+1}, Y_t^{\text{obs}}, \alpha, \beta_{\mu}, \beta_{\sigma}, \gamma, U_t, \epsilon_o) & \propto \exp\left(-Y_t \exp(\alpha + \beta_{\mu} X_{\mu} + \beta_{\sigma} X_{\sigma} + \gamma U_t)\right) \\
& \left(Y_t \exp(\alpha + \beta_{\mu} X_{\mu} + \beta_{\sigma} X_{\sigma} + \gamma U_t)\right)^{Y_{t+1}} \times \\ 
& \exp\left(-Y_t e^{\epsilon_o}\right) \left(Y_t e^{\epsilon_o}\right)^{Y_t^{\text{obs}}}
\end{split}
\end{equation}

This simplifies to:

\begin{equation}
\begin{split}
p(Y_t \mid Y_{t+1}, Y_t^{\text{obs}}, \alpha, \beta_{\mu}, \beta_{\sigma}, \gamma, U_t, \epsilon_o)  & \propto Y_t^{Y_{t+1} + Y_t^{\text{obs}}} \times \\
& \exp\left( -Y_t \left( \exp(\alpha + \beta_{\mu} X_{\mu} + \beta_{\sigma} X_{\sigma} + \gamma U_t) + e^{\epsilon_o} \right) \right)
\end{split}
\end{equation}

This is proportional to a Gamma distribution with shape
\(Y_{t+1} + Y_t^{\text{obs}}\) and rate
\(\exp(\alpha + \beta_{\mu} X_{\mu} + \beta_{\sigma} X_{\sigma} + \gamma U_t) + e^{\epsilon_o}\).
Therefore, the full conditional for \(Y_t\) is Gamma-distributed, and
there is a closed-form solution.

\subsubsection{\texorpdfstring{2. Full Conditional for
\(U_t\)}{2. Full Conditional for U\_t}}\label{full-conditional-for-u_t}

The likelihood for \(U_t\) comes from the Normal distribution and the
Poisson likelihood for \(Y_{t+1}\):

\begin{equation}
U_{t+1} \mid U_t, X_{\mu} \sim \mathcal{N}(\phi U_t + \eta X_{\mu}, \sigma_U^2)
\end{equation}

\begin{equation}
Y_{t+1} \mid Y_t, U_t \sim \text{Poisson}(Y_t \exp(\alpha + \beta_{\mu} X_{\mu} + \beta_{\sigma} X_{\sigma} + \gamma U_t))
\end{equation}

The joint likelihood is:

\begin{equation}
\begin{split}
p(U_t \mid U_{t+1}, Y_{t+1}, Y_t, \alpha, \beta_{\mu}, \beta_{\sigma}, \gamma, X_{\mu}, \sigma_U^2) & \propto \\
& \mathcal{N}(U_{t+1} \mid \phi U_t + \eta X_{\mu}, \sigma_U^2) \times \\
& \text{Poisson}(Y_{t+1} \mid Y_t \exp(\alpha + \beta_{\mu} X_{\mu} + \beta_{\sigma} X_{\sigma} + \gamma U_t))
\end{split}
\end{equation}

Substituting the likelihoods:

\begin{equation}
\begin{split}
p(U_t \mid U_{t+1}, Y_{t+1}, Y_t, \alpha, \beta_{\mu}, \beta_{\sigma}, \gamma, X_{\mu}, \sigma_U^2) &\propto \\
& \exp\left( -\frac{(U_{t+1} - (\phi U_t + \eta X_{\mu}))^2}{2 \sigma_U^2} \right) \times \\
& \exp\left( -Y_t \exp(\alpha + \beta_{\mu} X_{\mu} + \beta_{\sigma} X_{\sigma} + \gamma U_t) \right)
\end{split}
\end{equation}

This is a mixture of Gaussian and Poisson terms, which does not have a
closed-form solution. Numerical methods such as Metropolis-Hastings
would be used for approximation.

\subsubsection{\texorpdfstring{3. Full Conditional for
\(\alpha\)}{3. Full Conditional for \textbackslash alpha}}\label{full-conditional-for-alpha}

The prior for \(\alpha\) is \(\mathcal{N}(0, 10)\), and the full
conditional is:

\begin{equation}
p(\alpha \mid Y, X_{\mu}, X_{\sigma}, U) \propto \mathcal{N}(0, 10) \times \prod_{t=1}^{T} \text{Poisson}(Y_{t+1} \mid Y_t \exp(\alpha + \beta_{\mu} X_{\mu} + \beta_{\sigma} X_{\sigma} + \gamma U_t))
\end{equation}

Substituting the Poisson likelihood:

\begin{equation}
p(\alpha \mid Y, X_{\mu}, X_{\sigma}, U) \propto \exp\left( -\frac{\alpha^2}{2 \cdot 10^2} \right) \times \prod_{t=1}^{T} \exp\left( -Y_t \exp(\alpha + \beta_{\mu} X_{\mu} + \beta_{\sigma} X_{\sigma} + \gamma U_t) \right)
\end{equation}

This simplifies to:

\begin{equation}
p(\alpha \mid Y, X_{\mu}, X_{\sigma}, U) \propto \exp\left( -\frac{\alpha^2}{2 \cdot 10^2} - \sum_{t=1}^{T} Y_t \exp(\alpha + \beta_{\mu} X_{\mu} + \beta_{\sigma} X_{\sigma} + \gamma U_t) \right)
\end{equation}

This is a Gaussian-exponential mixture, and there is no closed-form
solution. Approximation methods (e.g., numerical optimization) would be
needed.

\subsubsection{\texorpdfstring{4. Full Conditional for
\(\beta_{\mu}\)}{4. Full Conditional for \textbackslash beta\_\{\textbackslash mu\}}}\label{full-conditional-for-beta_mu}

The prior for \(\beta_{\mu}\) is \(\mathcal{N}(0, 1)\), and the full
conditional is:

\begin{equation}
p(\beta_{\mu} \mid Y, X_{\mu}, X_{\sigma}, U) \propto \mathcal{N}(0, 1) \times \prod_{t=1}^{T} \text{Poisson}(Y_{t+1} \mid Y_t \exp(\alpha + \beta_{\mu} X_{\mu} + \beta_{\sigma} X_{\sigma} + \gamma U_t))
\end{equation}

Substituting the Poisson likelihood:

\begin{equation}
p(\beta_{\mu} \mid Y, X_{\mu}, X_{\sigma}, U) \propto \exp\left( -\frac{\beta_{\mu}^2}{2} \right) \times \prod_{t=1}^{T} \exp\left( -Y_t \exp(\alpha + \beta_{\mu} X_{\mu} + \beta_{\sigma} X_{\sigma} + \gamma U_t) \right)
\end{equation}

This leads to a Gaussian-exponential mixture, which does not have a
closed-form solution.

\subsubsection{\texorpdfstring{5. Full Conditional for
\(\beta_{\sigma}\)}{5. Full Conditional for \textbackslash beta\_\{\textbackslash sigma\}}}\label{full-conditional-for-beta_sigma}

The full conditional for \(\beta_{\sigma}\) follows the same structure
as for \(\beta_{\mu}\):

\begin{equation}
p(\beta_{\sigma} \mid Y, X_{\mu}, X_{\sigma}, U) \propto \mathcal{N}(0, 1) \times \prod_{t=1}^{T} \text{Poisson}(Y_{t+1} \mid Y_t \exp(\alpha + \beta_{\mu} X_{\mu} + \beta_{\sigma} X_{\sigma} + \gamma U_t))
\end{equation}

This does not have a closed-form solution either.

\subsubsection{\texorpdfstring{6. Full Conditional for
\(\gamma\)}{6. Full Conditional for \textbackslash gamma}}\label{full-conditional-for-gamma}

The full conditional for \(\gamma\) is:

\begin{equation}
p(\gamma \mid Y, X_{\mu}, X_{\sigma}, U) \propto \mathcal{N}(0, 1) \times \prod_{t=1}^{T} \text{Poisson}(Y_{t+1} \mid Y_t \exp(\alpha + \beta_{\mu} X_{\mu} + \beta_{\sigma} X_{\sigma} + \gamma U_t))
\end{equation}

This does not have a closed-form solution.

\subsubsection{\texorpdfstring{7. Full Conditional for
\(\phi\)}{7. Full Conditional for \textbackslash phi}}\label{full-conditional-for-phi}

The prior for \(\phi\) is \(\text{Uniform}(-1, 1)\), and the full
conditional is:

\begin{equation}
p(\phi \mid U, X_{\mu}) \propto \text{Uniform}(-1, 1) \times \prod_{t=1}^{T} \mathcal{N}(U_{t+1} \mid \phi U_t + \eta X_{\mu}, \sigma_U^2)
\end{equation}

This is a product of Gaussian distributions. The full conditional does
not have a closed-form solution but can be approximated numerically.

\subsubsection{\texorpdfstring{8. Full Conditional for
\(\eta\)}{8. Full Conditional for \textbackslash eta}}\label{full-conditional-for-eta}

Similarly, for \(\eta\):

\begin{equation}
p(\eta \mid U, X_{\mu}) \propto \mathcal{N}(0, 1) \times \prod_{t=1}^{T} \mathcal{N}(U_{t+1} \mid \phi U_t + \eta X_{\mu}, \sigma_U^2)
\end{equation}

This is a Gaussian-exponential mixture, and no closed-form solution
exists.

\subsubsection{\texorpdfstring{9. Full Conditional for
\(\sigma_U^2\)}{9. Full Conditional for \textbackslash sigma\_U\^{}2}}\label{full-conditional-for-sigma_u2}

For \(\sigma_U^2\), the prior is \(\text{Inverse-Gamma}(2, 1)\), and the
full conditional is:

\begin{equation}
p(\sigma_U^2 \mid U, X_{\mu}, \phi, \eta) \propto \text{Inverse-Gamma}(2, 1) \times \prod_{t=1}^{T} \mathcal{N}(U_{t+1} \mid \phi U_t + \eta X_{\mu}, \sigma_U^2)
\end{equation}

The full conditional is an \textbf{Inverse-Gamma distribution} with
updated shape and rate parameters. This has a \textbf{closed-form
solution}.

\subsubsection{\texorpdfstring{10. Full Conditional for
\(\sigma_o^2\)}{10. Full Conditional for \textbackslash sigma\_o\^{}2}}\label{full-conditional-for-sigma_o2}

For \(\sigma_o^2\), the prior is \(\text{Inverse-Gamma}(2, 1)\), and the
full conditional is:

\begin{equation}
p(\sigma_o^2 \mid Y^{\text{obs}}, Y) \propto \text{Inverse-Gamma}(2, 1) \times \prod_{t=1}^{T} \mathcal{N}(\log(Y_t^{\text{obs}} / Y_t) \mid 0, \sigma_o^2)
\end{equation}

This conditional is also \textbf{Inverse-Gamma} with parameters updated
based on the data. Hence, it has a \textbf{closed-form solution}.



\end{document}

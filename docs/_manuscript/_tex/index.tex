% Options for packages loaded elsewhere
\PassOptionsToPackage{unicode}{hyperref}
\PassOptionsToPackage{hyphens}{url}
\PassOptionsToPackage{dvipsnames,svgnames,x11names}{xcolor}
%
\documentclass[
  letterpaper,
  DIV=11,
  numbers=noendperiod]{scrartcl}

\usepackage{amsmath,amssymb}
\usepackage{iftex}
\ifPDFTeX
  \usepackage[T1]{fontenc}
  \usepackage[utf8]{inputenc}
  \usepackage{textcomp} % provide euro and other symbols
\else % if luatex or xetex
  \usepackage{unicode-math}
  \defaultfontfeatures{Scale=MatchLowercase}
  \defaultfontfeatures[\rmfamily]{Ligatures=TeX,Scale=1}
\fi
\usepackage{lmodern}
\ifPDFTeX\else  
    % xetex/luatex font selection
\fi
% Use upquote if available, for straight quotes in verbatim environments
\IfFileExists{upquote.sty}{\usepackage{upquote}}{}
\IfFileExists{microtype.sty}{% use microtype if available
  \usepackage[]{microtype}
  \UseMicrotypeSet[protrusion]{basicmath} % disable protrusion for tt fonts
}{}
\makeatletter
\@ifundefined{KOMAClassName}{% if non-KOMA class
  \IfFileExists{parskip.sty}{%
    \usepackage{parskip}
  }{% else
    \setlength{\parindent}{0pt}
    \setlength{\parskip}{6pt plus 2pt minus 1pt}}
}{% if KOMA class
  \KOMAoptions{parskip=half}}
\makeatother
\usepackage{xcolor}
\setlength{\emergencystretch}{3em} % prevent overfull lines
\setcounter{secnumdepth}{5}
% Make \paragraph and \subparagraph free-standing
\ifx\paragraph\undefined\else
  \let\oldparagraph\paragraph
  \renewcommand{\paragraph}[1]{\oldparagraph{#1}\mbox{}}
\fi
\ifx\subparagraph\undefined\else
  \let\oldsubparagraph\subparagraph
  \renewcommand{\subparagraph}[1]{\oldsubparagraph{#1}\mbox{}}
\fi


\providecommand{\tightlist}{%
  \setlength{\itemsep}{0pt}\setlength{\parskip}{0pt}}\usepackage{longtable,booktabs,array}
\usepackage{calc} % for calculating minipage widths
% Correct order of tables after \paragraph or \subparagraph
\usepackage{etoolbox}
\makeatletter
\patchcmd\longtable{\par}{\if@noskipsec\mbox{}\fi\par}{}{}
\makeatother
% Allow footnotes in longtable head/foot
\IfFileExists{footnotehyper.sty}{\usepackage{footnotehyper}}{\usepackage{footnote}}
\makesavenoteenv{longtable}
\usepackage{graphicx}
\makeatletter
\def\maxwidth{\ifdim\Gin@nat@width>\linewidth\linewidth\else\Gin@nat@width\fi}
\def\maxheight{\ifdim\Gin@nat@height>\textheight\textheight\else\Gin@nat@height\fi}
\makeatother
% Scale images if necessary, so that they will not overflow the page
% margins by default, and it is still possible to overwrite the defaults
% using explicit options in \includegraphics[width, height, ...]{}
\setkeys{Gin}{width=\maxwidth,height=\maxheight,keepaspectratio}
% Set default figure placement to htbp
\makeatletter
\def\fps@figure{htbp}
\makeatother

\KOMAoption{captions}{tableheading}
\makeatletter
\@ifpackageloaded{caption}{}{\usepackage{caption}}
\AtBeginDocument{%
\ifdefined\contentsname
  \renewcommand*\contentsname{Table of contents}
\else
  \newcommand\contentsname{Table of contents}
\fi
\ifdefined\listfigurename
  \renewcommand*\listfigurename{List of Figures}
\else
  \newcommand\listfigurename{List of Figures}
\fi
\ifdefined\listtablename
  \renewcommand*\listtablename{List of Tables}
\else
  \newcommand\listtablename{List of Tables}
\fi
\ifdefined\figurename
  \renewcommand*\figurename{Figure}
\else
  \newcommand\figurename{Figure}
\fi
\ifdefined\tablename
  \renewcommand*\tablename{Table}
\else
  \newcommand\tablename{Table}
\fi
}
\@ifpackageloaded{float}{}{\usepackage{float}}
\floatstyle{ruled}
\@ifundefined{c@chapter}{\newfloat{codelisting}{h}{lop}}{\newfloat{codelisting}{h}{lop}[chapter]}
\floatname{codelisting}{Listing}
\newcommand*\listoflistings{\listof{codelisting}{List of Listings}}
\makeatother
\makeatletter
\makeatother
\makeatletter
\@ifpackageloaded{caption}{}{\usepackage{caption}}
\@ifpackageloaded{subcaption}{}{\usepackage{subcaption}}
\makeatother
\ifLuaTeX
  \usepackage{selnolig}  % disable illegal ligatures
\fi
\usepackage{bookmark}

\IfFileExists{xurl.sty}{\usepackage{xurl}}{} % add URL line breaks if available
\urlstyle{same} % disable monospaced font for URLs
\hypersetup{
  pdftitle={Estimation of the effect of climate on infectious disesaes},
  colorlinks=true,
  linkcolor={blue},
  filecolor={Maroon},
  citecolor={Blue},
  urlcolor={Blue},
  pdfcreator={LaTeX via pandoc}}

\title{Estimation of the effect of climate on infectious disesaes}
\author{}
\date{}

\begin{document}
\maketitle

\renewcommand*\contentsname{Table of contents}
{
\hypersetup{linkcolor=}
\setcounter{tocdepth}{3}
\tableofcontents
}
\section{Motivating Question}\label{motivating-question}

If climate change (through some sort of variable such as temperature or
precipitation) is affecting the number of cases of infectious diseasse,
it is an outstanding question how strong this effect must be to identify
it from some background autocorrelated value.

How strong does the signal of climate change need to be to detect it?

\section{Data \& model}\label{data-model}

Assuming we have some data on observed cases of a given infectious
disease. The relationship between those observed cases and actual cases
is a state process with some observation error, \(\epsilon_o\). Cases
themselves are now given as a state space model where the number of
cases at time \(t+1\) are driven by the effect of both temperature
variance (consistent through time) and mean temperature (increasing
through time), as well as an unobserved driver that is correlated
through time with the mean temperature.

Let: - \(Y_t\) represent the true number of cases at time \(t\). -
\(Y_t^{\text{obs}}\) represent the observed number of cases at time
\(t\), which includes observational error \(\epsilon_o\). - \(X_{\mu}\)
represent the mean temperature at time \(t\), and let \(X_{\sigma}\)
represent the temperature variance, assumed constant over time. -
\(U_t\) represent the unobserved driver correlated with mean temperature
\(X_{\mu}\).

\subsection{State Process (Evolution)
Equation}\label{state-process-evolution-equation}

For the true cases \(Y_t\), we specify that the cases at time \(t+1\)
are driven by the temperature effects and an unobserved driver \(U_t\)
as follows:

\begin{equation}
Y_{t+1} \mid Y_t, X_{\mu}, U_t \sim \text{Poisson} \left(Y_t \exp \left( \alpha + \beta_{\mu} X_{\mu} + \beta_{\sigma} X_{\sigma} + \gamma U_t \right)\right)
\end{equation}

where: - \(\alpha\) is an intercept term. - \(\beta_{\mu}\) is the
effect of mean temperature \(X_{\mu}\) on the cases. -
\(\beta_{\sigma}\) is the effect of temperature variance \(X_{\sigma}\).
- \(\gamma\) captures the effect of the unobserved driver \(U_t\) on the
cases.

\subsection{Latent Driver Process}\label{latent-driver-process}

We assume that the unobserved driver \(U_t\) has temporal correlation
and is influenced by the mean temperature:

\begin{equation}
U_{t+1} \mid U_t, X_{\mu} \sim \mathcal{N}(\phi U_t + \eta X_{\mu}, \sigma_U^2)
\end{equation}

where: - \(\phi\) is an autoregressive parameter governing the temporal
correlation of \(U_t\). - \(\eta\) is the strength of the correlation
between \(U_t\) and \(X_{\mu}\). - \(\sigma_U^2\) is the variance of
\(U_t\).

\subsection{Observation Equation}\label{observation-equation}

The observed cases \(Y_t^{\text{obs}}\) are related to the true cases
\(Y_t\) with observation error \(\epsilon_o\):

\begin{equation}
Y_t^{\text{obs}} \mid Y_t \sim \text{Poisson}(Y_t e^{\epsilon_o})
\end{equation}

with \(\epsilon_o \sim \mathcal{N}(0, \sigma_o^2)\), where
\(\sigma_o^2\) is the variance of the observation error.

\subsection{Priors}\label{priors}

The priors for the parameters could be specified as follows:

\begin{subequations}
\begin{align}
\alpha &\sim \mathcal{N}(0, 10) \\ 
\beta_{\mu} &\sim \mathcal{N}(0, 1) \\
\beta_{\sigma} &\sim \mathcal{N}(0, 1) \\ 
\gamma &\sim \mathcal{N}(0, 1)
\end{align}
\end{subequations}

\begin{subequations}
\begin{align}
\phi &\sim \text{Uniform}(-1, 1) \\ 
\eta &\sim \mathcal{N}(0, 1)  \\ 
\sigma_U^2 &\sim \text{Inverse-Gamma}(2, 1) \\
\sigma_o^2 &\sim \text{Inverse-Gamma}(2, 1) 
\end{align}
\end{subequations}



\end{document}

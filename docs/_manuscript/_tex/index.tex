% Options for packages loaded elsewhere
\PassOptionsToPackage{unicode}{hyperref}
\PassOptionsToPackage{hyphens}{url}
\PassOptionsToPackage{dvipsnames,svgnames,x11names}{xcolor}
%
\documentclass[
  letterpaper,
  DIV=11,
  numbers=noendperiod]{scrartcl}

\usepackage{amsmath,amssymb}
\usepackage{iftex}
\ifPDFTeX
  \usepackage[T1]{fontenc}
  \usepackage[utf8]{inputenc}
  \usepackage{textcomp} % provide euro and other symbols
\else % if luatex or xetex
  \usepackage{unicode-math}
  \defaultfontfeatures{Scale=MatchLowercase}
  \defaultfontfeatures[\rmfamily]{Ligatures=TeX,Scale=1}
\fi
\usepackage{lmodern}
\ifPDFTeX\else  
    % xetex/luatex font selection
\fi
% Use upquote if available, for straight quotes in verbatim environments
\IfFileExists{upquote.sty}{\usepackage{upquote}}{}
\IfFileExists{microtype.sty}{% use microtype if available
  \usepackage[]{microtype}
  \UseMicrotypeSet[protrusion]{basicmath} % disable protrusion for tt fonts
}{}
\makeatletter
\@ifundefined{KOMAClassName}{% if non-KOMA class
  \IfFileExists{parskip.sty}{%
    \usepackage{parskip}
  }{% else
    \setlength{\parindent}{0pt}
    \setlength{\parskip}{6pt plus 2pt minus 1pt}}
}{% if KOMA class
  \KOMAoptions{parskip=half}}
\makeatother
\usepackage{xcolor}
\setlength{\emergencystretch}{3em} % prevent overfull lines
\setcounter{secnumdepth}{5}
% Make \paragraph and \subparagraph free-standing
\ifx\paragraph\undefined\else
  \let\oldparagraph\paragraph
  \renewcommand{\paragraph}[1]{\oldparagraph{#1}\mbox{}}
\fi
\ifx\subparagraph\undefined\else
  \let\oldsubparagraph\subparagraph
  \renewcommand{\subparagraph}[1]{\oldsubparagraph{#1}\mbox{}}
\fi


\providecommand{\tightlist}{%
  \setlength{\itemsep}{0pt}\setlength{\parskip}{0pt}}\usepackage{longtable,booktabs,array}
\usepackage{calc} % for calculating minipage widths
% Correct order of tables after \paragraph or \subparagraph
\usepackage{etoolbox}
\makeatletter
\patchcmd\longtable{\par}{\if@noskipsec\mbox{}\fi\par}{}{}
\makeatother
% Allow footnotes in longtable head/foot
\IfFileExists{footnotehyper.sty}{\usepackage{footnotehyper}}{\usepackage{footnote}}
\makesavenoteenv{longtable}
\usepackage{graphicx}
\makeatletter
\def\maxwidth{\ifdim\Gin@nat@width>\linewidth\linewidth\else\Gin@nat@width\fi}
\def\maxheight{\ifdim\Gin@nat@height>\textheight\textheight\else\Gin@nat@height\fi}
\makeatother
% Scale images if necessary, so that they will not overflow the page
% margins by default, and it is still possible to overwrite the defaults
% using explicit options in \includegraphics[width, height, ...]{}
\setkeys{Gin}{width=\maxwidth,height=\maxheight,keepaspectratio}
% Set default figure placement to htbp
\makeatletter
\def\fps@figure{htbp}
\makeatother

\KOMAoption{captions}{tableheading}
\makeatletter
\@ifpackageloaded{caption}{}{\usepackage{caption}}
\AtBeginDocument{%
\ifdefined\contentsname
  \renewcommand*\contentsname{Table of contents}
\else
  \newcommand\contentsname{Table of contents}
\fi
\ifdefined\listfigurename
  \renewcommand*\listfigurename{List of Figures}
\else
  \newcommand\listfigurename{List of Figures}
\fi
\ifdefined\listtablename
  \renewcommand*\listtablename{List of Tables}
\else
  \newcommand\listtablename{List of Tables}
\fi
\ifdefined\figurename
  \renewcommand*\figurename{Figure}
\else
  \newcommand\figurename{Figure}
\fi
\ifdefined\tablename
  \renewcommand*\tablename{Table}
\else
  \newcommand\tablename{Table}
\fi
}
\@ifpackageloaded{float}{}{\usepackage{float}}
\floatstyle{ruled}
\@ifundefined{c@chapter}{\newfloat{codelisting}{h}{lop}}{\newfloat{codelisting}{h}{lop}[chapter]}
\floatname{codelisting}{Listing}
\newcommand*\listoflistings{\listof{codelisting}{List of Listings}}
\makeatother
\makeatletter
\makeatother
\makeatletter
\@ifpackageloaded{caption}{}{\usepackage{caption}}
\@ifpackageloaded{subcaption}{}{\usepackage{subcaption}}
\makeatother
\ifLuaTeX
  \usepackage{selnolig}  % disable illegal ligatures
\fi
\usepackage{bookmark}

\IfFileExists{xurl.sty}{\usepackage{xurl}}{} % add URL line breaks if available
\urlstyle{same} % disable monospaced font for URLs
\hypersetup{
  pdftitle={Estimation of the effect of climate on infectious disesaes},
  colorlinks=true,
  linkcolor={blue},
  filecolor={Maroon},
  citecolor={Blue},
  urlcolor={Blue},
  pdfcreator={LaTeX via pandoc}}

\title{Estimation of the effect of climate on infectious disesaes}
\author{}
\date{}

\begin{document}
\maketitle

\renewcommand*\contentsname{Table of contents}
{
\hypersetup{linkcolor=}
\setcounter{tocdepth}{3}
\tableofcontents
}
\section{Motivating Question}\label{motivating-question}

If climate change (through some sort of variable such as temperature or
precipitation) is affecting the number of cases of infectious disease,
it is an outstanding question how strong this effect must be to identify
it from some background autocorrelated value. It is worth asking how
correlated a confounding background process can be before we cannot
recover a value of interest.

\section{Data \& model}\label{data-model}

Assuming we have some data on observed cases of a given infectious
disease. The relationship between those observed cases and actual cases
is a state process with some observation error, \(\epsilon_o\). Cases
themselves are now given as a state space model where the number of
cases at time \(t+1\) are driven by the effect of both temperature
variance (consistent through time) and mean temperature (increasing
through time), as well as an unobserved driver that is correlated
through time with the mean temperature. We can simplify this for a first
pass and assume that there's no error in measurement. With this, then we
say:

Let:

\begin{itemize}
\tightlist
\item
  \(Y_t\) represent the observed number of cases at time \(t\).
\item
  \(X_{\mu,t}\) represent the mean temperature at time \(t\), and let
  \(X_{\sigma,t}\) represent the temperature variance, assumed constant
  over time.
\item
  \(U_t\) represent the unobserved driver correlated with mean
  temperature \(X_{\mu,t}\).
\end{itemize}

\subsection{Model Equation}\label{model-equation}

We can consider the model of observed cases where the number of observed
cases at some future timepoint \(t+1\) is given as

\begin{equation}
Y_{t+1} = \beta_0 + \beta_1 X_{\mu,t} + \beta_2 X_{\sigma, t} + U_t
\end{equation}

\subsection{\texorpdfstring{Likelihood for
\(Y_{t+1}\)}{Likelihood for Y\_\{t+1\}}}\label{likelihood-for-y_t1}

Assume the response \(Y_{t+1}\)follows a Negative Binomial distribution
with mean \(\lambda_t\)and dispersion \(\phi\): \begin{equation}
Y_{t+1} \sim \text{NegBin}(\lambda_t, \phi), \quad \lambda_t = \exp(\beta_0 + \beta_1 X_{\mu,t} + \beta_2 X_{\sigma,t} + U_t).
\end{equation}

\subsection{\texorpdfstring{Unobserved Process
\(U_t\)}{Unobserved Process U\_t}}\label{unobserved-process-u_t}

The unobserved process \(U_t\)is modeled as a latent variable, for
example, using a Gaussian random walk or autoregressive process:
\begin{equation}
U_t \sim \mathcal{N}(\rho U_{t-1}, \tau^2),
\end{equation} where \(\rho\) controls the correlation structure, and
\(\tau^2\) is the process variance.

\subsection{Prior Distributions}\label{prior-distributions}

Assign priors to the parameters: \begin{equation}
\beta_0, \beta_1, \beta_2 \sim \mathcal{N}(0, 10), \quad \phi \sim \text{Gamma}(1, 1),
\end{equation} and \begin{equation}
\rho \sim \mathcal{Beta}(2, 2), \quad \tau^2 \sim \text{Inverse-Gamma}(2, 1).
\end{equation}

\subsection{Full Model}\label{full-model}

The full model is given by: \begin{align*}
Y_{t+1} &\sim \text{NegBin}(\lambda_t, \phi), \\
\lambda_t &= \exp(\beta_0 + \beta_1 X_{\mu,t} + \beta_2 X_{\sigma,t} + U_t), \\
U_t &\sim \mathcal{N}(\rho U_{t-1}, \tau^2), \quad \text{for } t = 2, \ldots, T, \\
U_1 &\sim \mathcal{N}(0, \tau^2).
\end{align*}



\end{document}
